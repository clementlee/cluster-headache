\documentclass[11pt]{article}
\usepackage{libertine}
\usepackage{amsmath}
\usepackage{amssymb}
\usepackage{microtype}

\title{\huge Optimal Lobotomy}
\author{Clement Lee\\\emph{advised by Jianxiong Xiao}}
\date{16 Feb 2016}
\begin{document}
%Title. Include your name, class, project title, and advisor.
\maketitle

%Motivation and Goal. Give a high-level introduction to your topic area and state specifically what problem you will be addressing.  You should clearly state the goal of your project with a sentence beginning “The goal of my project is …”. Explain why that goal is important and/or interesting, perhaps with a description of applications or people enabled by achieving it.
\section{Motivation and Goal}
The recent and astronomical rise of deep learning algorithmsapplied to a variety of datasets has been built on the foundation of significant boosts to computational power, especially with the advent of GPGPU techniques.

%Problem Background and Related Work. Place your project in context of prior work.  Give some background of what has been done before to achieve your stated goal. Include citations to closely related academic papers and/or list commercial products targeted at the same goal.  Finish this section with a brief explanation of the problem unsolved by previous work that will be addressed by your project.
\section{Problem Background and Related Work}
Of primary importance to this work is the paper \emph{Optimal Brain Damage} \cite{lecun1989optimal} which describes an algorithm to reduce the size of a feedforward neural network by removing elements which contribute little to its classification ability.


%Approach. Provide a concise description of the key idea underlying your approach to achieving the stated goal.  Provide an argument of why your approach is a good idea – i.e., why it can achieve the stated goal where others have not.
\section{Approach}


%Plan. Describe the steps you plan to take and/or the issues you plan to address during the execution of your project.  What data sets will you have to acquire?  What algorithms will you have to develop?  What theorems will you have to prove?  etc.  For the steps that are non-trivial, provide a brief description of the issue, options, and planned approach.    Please indicate any particularly risky aspects of the project and discuss contingencies in case they do not go as planned.
\section{Plan}

%Evaluation. Describe the methodology you plan to use to evaluate how well your project has achieved the stated goal.  Be specific.  What data will you use?  What test will you run?  What quantitative metric will you use to measure success?  etc.  This is a very important and often over-looked aspect of a project plan – please think about it before finalizing your project selection.
\section{Evaluation}

\bibliographystyle{acm}
\bibliography{ref}
\end{document}
